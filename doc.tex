% Created 2016-02-23 mar 05:01
\documentclass[11pt]{article}
\usepackage[utf8]{inputenc}
\usepackage[T1]{fontenc}
\usepackage{fixltx2e}
\usepackage{graphicx}
\usepackage{longtable}
\usepackage{float}
\usepackage{wrapfig}
\usepackage{rotating}
\usepackage[normalem]{ulem}
\usepackage{amsmath}
\usepackage{textcomp}
\usepackage{marvosym}
\usepackage{wasysym}
\usepackage{amssymb}
\usepackage{hyperref}
\tolerance=1000
\author{Darias}
\date{\today}
\title{doc}
\hypersetup{
  pdfkeywords={},
  pdfsubject={},
  pdfcreator={Emacs 24.5.1 (Org mode 8.2.10)}}
\begin{document}

\maketitle
\tableofcontents

\section{asdasdasdasdasdasd}
\label{sec-1}

\subsection{asdasdasd}
\label{sec-1-1}


\begin{verbatim}
int main(void){
cout<<"turbo ñame"<<endl;
}
\end{verbatim}
sdf

sdf

sf

sdf

df

sdf

sdf



\begin{center}
\begin{tabular}{rrlllll}
Estado & [jJ]* & : & /d & [*,=] & Otro caso & Consideraciones\\
\hline
0 & 2 & 3 & Error & Error & Guardar en buffer, pasa a estado 1 & Buff puede contener un opcode o una etiqueta de entrada\\
1 & 2 & 3 & 0 & 4 & Error & Ya hemos averiguado qué es el buffer\\
2 & Error & Error & Error & Error & Pasa a estado 0 & El token contiene la etiqueta de salida, Fin instrucción\\
3 & 2 & Error & Error & Error & Pasa al estado 5 & El token contiene el opcode\\
4 & Error & Error & 0 & Error & Error & Fin instrucción\\
5 & 2 & Error & 0 & 4 & Error & \\
\end{tabular}
\end{center}



dasñśaññdada asdllañks ñl

asdasd
% Emacs 24.5.1 (Org mode 8.2.10)
\end{document}
